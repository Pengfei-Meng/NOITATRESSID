%%%%%%%%%%%%%%%%%%%%%%%%%%%%%%%%%%%%%%%%%%%%%%%%%%%%%%%%%%%%%%%%%%% 
%                                                                 %
%                            CHAPTER SEVEN                         %
%                                                                 %
%%%%%%%%%%%%%%%%%%%%%%%%%%%%%%%%%%%%%%%%%%%%%%%%%%%%%%%%%%%%%%%%%%% 
 
\chapter{CONCLUSIONS AND RECOMMENDATIONS}\label{chap:con}

\section{Conclusions}
When solving PDE-constrained design optimization problems, many practitioners
favor a reduced-space formulation in which the state variables are implicit
functions of the design variables.  However, conventional reduced-space
algorithms are not well suited to problems with many state-based constraints, 
such as structural stress constraints, 
because these algorithms require the contraint Jacobian explicilty, which
requires an adjoint for each row.

In this work we have developed a matrix-free algorithm to handle reduced-space
PDE-constrained optimization with state-based constraints.  To cope with
possibly indefinite or negative-definite Hessians, we follow the zero curve of a homotopy map using a
predictor-corrector algorithm.  The tangent predictor and Newton corrector update linear systems are
solved using a Krylov iterative method.  To precondition the Krylov solver, we
proposed a low-rank, approximate singular-value decomposition of the Schur
complement of the KKT matrix with respect to the slacks and inequality multipliers.  All
the components of the algorithm - the predictor-corrector scheme, the Krylov
method, and the preconditioner - require only matrix-vector products and,
thus, avoid the need to compute the constraint Jacobian or Lagrangian Hessian
explicitly.

The numerical experiments suggest that the matrix-free algorithm is effective.
In particular, the results indicate that the algorithm can avoid stationary
points that are not minimizers and that it scales well with the number of design
variables.  On a synthetic quadratic optimization problem the proposed algorithm
was shown to be competitive with a state-of-the-art (matrix-explicit) active-set
SQP algorithm.  Furthermore, when applied to a difficult structural optimization
problem the algorithm was able to satisfy the optimality and feasibility
criteria whereas the matrix-explicit SQP algorithm failed to converge. 

%The proposed algorithm indirectly uses the homotopy term as a globalization method, 
%thus is avoid of line-search or trust-region globalization techniques. 
%It is straightforward and clean to implement. 

\section{Recommendations}

\begin{enumerate}

\item The numerical experiments suggest that, like PDEs, there are no preconditioners that work universally well for all optimization problems.  This is especially true for "matrix-free" preconditioners.  Therefore, it is recommended that specialized preconditioners be investigated for different types of reduced-space PDE-constrained optimization problems; for example, unique preconditioners for structural-stress constraints that scale with the problem size versus aerodynamic state-based constraints that are of fixed number.

\item During the course of this research, it was found that distinguishing between nonlinear and linear constraints is important for the construction of the matrix-free preconditioner. 
Unfortunately, given the current API in Kona, the current implementation of the preconditioner must apply the Lanczos SVD method to approximate the whole inequality block in~\ref{eq:svd}, 
After separating the linear constraints with nonlinear constraints, the SVD
approximation will be only applied to the nonlinear part; the explicit linear Jacobians could be stored and used in the same way as the conventional optimization methods. 
The explicit storage of the linear constraint Jacobian would mean that the algorithm is only "matrix-free" with respect to the nonlinear constraints, but the additional storage cost may be outweighed by the reduced computational cost. 

\item While the homotopy-based globalization is effective, it often requires very small predictor steps and tight corrector tolerances to avoid local maximizers and saddle points.  Unfortunately, following the zero curve this closely is computationally costly.  It is recommended that other globalization methods be considered, including, perhaps, using no globalization but adding an a posterior check on the curvature of the Hessian. 

\item Once the algorithmic recommendations have been implemented, the aerodynamic shape optimization problem should be revisited using the RANS equations and a transonic flow.  This will provide a more challenging and realistic test for the optimization algorithm. 

\item In addition to comparisons with SNOPT, the proposed algorithms should be benchmarked against different types of optimization algorithms.  In particular, Kona could be compared to the interior-point algorithm implemented in the IPOPT package.

\item The proposed homotopy RSNK algorithm has many parameters.  While performance is insensitive to many of these parameters, there are some parameters that must be chosen carefully.  To help the user, robust and automated procedures should be developed in order to set these parameters.

\end{enumerate}


%\item  Previously, Lyu et al. presented a comprehensive set of results on aerodynamic shape optimization of the NASA CRM wing based on a RANS model with large numbers of shape variables using SNOPT~\cite{2015lyu_crm}. Although the study in this thesis at this stage is based on the Euler model and a coarser mesh grid than those in~\cite{2015lyu_crm}, the long term objective is to run on RANS model with a fine mesh.   

%\item  As described in~\ref{sec:cuternm}, CUTEr separates 
%the bound constraints on the design variables from other types of constraints, providing $\text{bl}$ 
%and $\text{bu}$. For other types of constraints, CUTEr align them into a vector using $c(x)$, 
%together with the bound $\text{cl}$ and $\text{cu}$. SNOPT's solver interface exactly follows this 
%pattern, and the user only need to provide $\text{bl}$ and $\text{bu}$,  $\text{cl}$ and $\text{cu}$ 
%after defining $c(x)$. In contrast, in Kona, the bound constraints are not separated from other 
%types of constraints; inequality constraints are defined on only one side.  
%So if there are both a lower and an upper bound constraint for a certain 
%variable, $\text{bl} \leq x_i \leq \text{bl}$, it has to be defined as two inequality constraints 
%$x_i - \text{bl} \geq 0 $ and $\text{bu} - x_i \geq 0 $. Perhaps in the future the solver interface of Kona 
%can be changed in a simpler way?  
%The current preconditioner is using the Lanczos SVD method to approximate the whole 
%inequality Jacobian block in~\ref{eq:svd}, irrespective of linear or nonlinear constraints. 
%That is a wasteful practice for not making use of the linear constraint Jacobians fully, but computing it each time 
%and only use its matrix-vector product. 

%The Lanczos SVD preconditioner will be more effective in approximating only the nonlinear inequality constraint part. Take the ASO problem for example, 
%the linear constraints are consist of the geometric constraints, whose Jacobians are cheap 
%to evaluate and only depend on the design variables. The nonlinear aerodynamic constraints are the part that benefit the most from matrix-vector product approximations using second-order adjoints. 
%One potential counter argument could be that since Kona is branded as the ``matrix-free '' preconditioner, 
%wouldn't it be self-deprecating to use explicit Jacobians for the linear constraints? We could be a little 
%utilitarian and say that ``matrix-free '' is only used when necessary, for those expensive nonlinear 
%constraints dependent on state variables. 



%%%%%%%%%%%%%%%%%%%%%%%%%%%%%%%%%%%%

%%% Local Variables: 
%%% mode: latex
%%% TeX-master: t
%%% End: 
