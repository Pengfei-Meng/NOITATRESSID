%%%%%%%%%%%%%%%%%%%%%%%%%%%%%%%%%%%%%%%%%%%%%%%%%%%%%%%%%%%%%%%%%%% 
%                                                                 %
%                            CHAPTER ONE                          %
%                                                                 %
%%%%%%%%%%%%%%%%%%%%%%%%%%%%%%%%%%%%%%%%%%%%%%%%%%%%%%%%%%%%%%%%%%% 
 
\chapter{SOLVING REDUCED-SPACE PDE-CONSTRAINED OPTIMIZATION}
 
\section{PDE-constrained Optimization Problem}\label{sec:pde_mot}
A generic PDE-constrained optimization problem can be stated as
\begin{equation}\label{eq:gen1}
\begin{aligned}
\underset{x,u}{\text{min}} \quad &f(x, u) &\\
\text{subject to} \quad &  h(x,u) &= 0  \\
 &  g(x,u) &\geq 0  \\
\text{governed by} \quad &  \mathcal{F}(x, u) &= 0, \\
\end{aligned}
\end{equation}
where $x \in \mathbb{R}^n, u \in \mathbb{R}^v$ are the design and state vectors respectively, and $f: \mathbb{R}^n \rightarrow \mathbb{R}, h: \mathbb{R}^n \rightarrow \mathbb{R}^l, g:\mathbb{R}^n \rightarrow \mathbb{R}^m$ are the objective, equality and inequality constraints respectively. We assume that $f$, $h$ and $g$ have continuous second derivatives. 

The challenge in solving \eqref{eq:gen1} is that general-purpose gradient-based optimization algorithms \cite{Nocedal2006NO, Byrd:1999:IPA:588897.589167,gill:2002} require the total derivative of the objective and constraints with respect to the design variables, and each of these total derivatives requires the solution of an $v\times v$ linear system, \ie a discretized PDE.  For example, the total derivative of the $i$th constraint $g_i$ is given by
\begin{align}
\frac{d}{dx}\left(g_i\right) &= \frac{\partial}{\partial x}\left(g_i\right) + \psi ^T \left[ \frac{\partial}{\partial x}\mathcal{F} \right] \\
\intertext{where $\psi \in \mathbb{R}^v$ is the adjoint which is governed by the discretized PDE,}
\left[ \frac{\partial}{\partial u}\mathcal{F} \right]^T \psi &= - \frac{\partial}{\partial u}\left(g_i\right) 
\end{align}
which is the adjoint equation \cite{Jameson03aerodynamicshape}. The computational cost of solving the adjoint equation is equivalent to that of the PDEs governing equation. Clearly, if the number of constraints $m$ is sufficiently large, it will be prohibitively expensive to compute all the total derivatives for optimization as each one entails an adjoint. 

To solve \eqref{eq:gen1}, the Lagrangian formulation is:
\begin{equation}\label{eq:lag}
\mathcal{L}(x, u, \psi, s, \lambda_h, \lambda_g) = f(x,u) + \lambda_h^T h(x, u) + \lambda_g^T (g(x,u)-s) + \psi^T \mathcal{F}(x,u),
\end{equation} 
where $s \in \mathbb{R}^m$ is for transforming the inequality constraints into equality ones, and $\lambda_h \in  \mathbb{R}^l$,  $\lambda_g \in  \mathbb{R}^m$ are the Lagrangian multiplier vectors for equality and inequality constraints. 

The first-order optimality conditions, or the \textit{Karush-Kuhn-Tucker} (KKT) conditions for \eqref{eq:gen1} are derived by taking partial derivatives of 
\eqref{eq:lag} with respect to all the unknown variables,
\begin{equation}\label{eq:kktcond}
\begin{aligned}
\nabla_x \mathcal{L} &= \nabla_x f + \lambda_h^T \nabla_x h + \lambda_g^T \nabla_x g + \psi^T \nabla_x\mathcal{F} = 0, \\
\nabla_u \mathcal{L} &= \nabla_u f + \lambda_h^T \nabla_u h + \lambda_g^T \nabla_u g + \psi^T \nabla_u\mathcal{F} = 0, \\
\nabla_{\psi} \mathcal{L} &= \mathcal{F} = 0, \\
\nabla_{\lambda_h} \mathcal{L} &= h = 0, \\
\nabla_{\lambda_g} \mathcal{L} &= g - s = 0, \\
-\mathcal{S} \Lambda_g e &= 0,\\
s \geq 0, &\quad \lambda_g \leq 0. \\
\end{aligned}
\end{equation}

The big nonlinear systems of equations \eqref{eq:kktcond} can be solved in either the full space or the reduced space.  Full-space methods~\cite{DBLP:journals/siamsc/BirosG05,DBLP:journals/siamsc/BirosG05a,haber:2001} solve all the unknowns in \eqref{eq:kktcond} simultaneously. The resulted KKT system is large, more than two times the number of state variables, indefinite and ill-conditioned. During optimization iterations, the state equation $\mathcal{F} =0$ and adjoint equation $\nabla_u \mathcal{L} = 0$ do not need to be solved exactly, avoiding the computational expense of tightly converging the PDE and adjoint equation residuals; however, this is also a potential disadvantage in practical engineering problems, because, if the optimization fails to converge, the intermediate solution may not be feasible with respect to the physics.  Furthermore, for highly nonlinear PDEs, \eg gas dynamics with shocks and boundary layers, practitioners have developed specialized globalization strategies that may be difficult to take advantage of in general-purpose full-space optimization algorithms.

In contrast, reduced-space algorithms treat the states $u$ and the adjoints $\psi$ as implicit functions of the design variables through $\mathcal{F}(x,u(x)) = 0$, $\nabla_u \mathcal{L} = 0$. The reduced KKT condition is formulated as follows, 
\begin{equation}\label{eq:opt00x}
 \begin{gathered}
    F(x,s,\lambda_h, \lambda_g) \equiv 
    \begin{bmatrix}
\nabla_x f + \lambda_h^T \nabla_x h + \lambda_g^T \nabla_x g\\
-\mathcal{S} \Lambda_g e\\
h  \\
g - s 
\end{bmatrix} =0,\\
\text{subject to} \quad s_i \geq 0, \quad \text{and} \quad \lambda_{gi} \leq 0. \\
\end{gathered}
\end{equation}

where $s \in \mathbb{R}^m$ denotes the slack variables, $q \equiv (x^T, s^T,
\lambda_h^T, \lambda_g^T) \in \mathbb{R}^{N}$ is a vector of all the unknowns,
and $F :\mathbb{R}^{N} \rightarrow \mathbb{R}^{N}$ is the vector-valued residual
of the KKT conditions excluding the inequalities on $s$ and $\lambda_g$.  For
convenience, we have also introduced $e = [1,1,\ldots,1]^T$ and the diagonal
matrices
\begin{equation*}
  \mat{S} = \mydiag\left(s_1,s_2,\ldots,s_m\right),\qquad\text{and}\qquad
  \mat{\Lambda_g} = \mydiag\left(\lambda_{g1}, \lambda_{g2}, \ldots, \lambda_{gm}\right).
\end{equation*}


Therefore, the resulted KKT system is much smaller, and it can make use of existing PDE solvers and adjoint solvers, maintaining modularity. Reduced-space methods have been successfully implemented in unconstrained problems and IDF problems. The unconstrained version of \eqref{eq:opt00x} can be solved efficiently using a Newton-Krylov (NK) algorithm applied to the first-order optimality conditions; see, for example, \cite{akcelik:2006, Heinkenschloss:1999:IOA, hinze2010optimization,borzi:2011}. NK optimization algorithms have also shown promise for equality-constrained optimization in the reduced space, because they do not require the constraint Jacobian to be form explicitly and, thus, avoid the scaling issue described earlier.  For instance, \cite{dener:idf2017} applied a matrix-free NK algorithm to a class of equality-constrained optimization problems that arise in multidisciplinary design optimization and would otherwise be intractable with conventional matrix-based algorithms.

Motivated by its success in the unconstrained and equality-constrained cases, we would like to extend the Newton Krylov methodology to more general, inequality constrained problems, to solve \eqref{eq:opt00x}. 

\section{Inexact-Newton Methods}

An alternative to using conventional (matrix-based) optimization algorithms or
constraint aggregation is to apply inexact-Newton methods~\cite{dembo:1982},
also know as truncated-Newton methods in the optimization
literature~\cite{nash:2000} to solve \eqref{eq:opt00x}.  

With the exception of the bounds on $s$ and $\lambda_g$, the KKT conditions
\eqref{eq:opt00x} form a set of nonlinear algebraic equations, $F(q)=0$.  These
equations can be solved, in principle, using Newton iterations of the form
\begin{equation}
(\nabla_q F) \Delta q^{(k)} = -F(q^{(k)}), \label{eq:Newton}
\end{equation}
where $q^{(k)}$ is the solution at the $k$th iteration and $\Delta q^{(k)} =
q^{(k+1)} - q ^{(k)}$ is the solution update.  Solving~\eqref{eq:Newton} exactly
can be inefficient during early Newton iterates when the linear model is not a
good approximation to $F(q)=0$.  Instead, truncated- and inexact-Newton methods
find approximate solutions to \eqref{eq:Newton} that, for example, satisfy the
inexact-Newton condition
\begin{equation}
  \left\| (\nabla_q F) \Delta q^{(k)} + F(q^{(k)}) \right\| \leq \eta_k \left\| F(q^{(k)}) \right\|, \label{eq:inexact_Newton}
\end{equation}
for some parameter $\eta_k \in (0,1)$.

There has been considerable success applying inexact-Newton methods to
unconstrained optimization problems; see \cite{nash:2000} and the references
therein.  On the other hand, inexact-Newton methods for general (nonconvex)
constrained problems are much less common.  Some notable exceptions include the
efforts by Byrd and colleagues~\cite{byrd:2008, byrd:2010} and by Heinkenschloss
and Ridzal~\cite{heinkenschloss:2014}; however, these algorithms make
assumptions regarding the structure of the problem that favor full-space
formulations, and our experience applying them to reduced-space PDE-constrainted
optimization has been disappointing.

Applying Newton's method to \eqref{eq:opt00x}, the KKT system, also called primal-dual system is obtained:
\begin{equation}\label{eq:kkt0}
\begin{bmatrix} \nabla_{xx} \mathcal{L} & 0 & \mathcal{A}_{h}^T & \mathcal{A}_{g}^T \\
    0 & -\Lambda_g & 0 & -\mathcal{S} \\
    \mathcal{A}_{h} & 0 & 0 & 0 \\
    \mathcal{A}_{g} & -\mathcal{I} & 0 & 0 
    \end{bmatrix}
    \begin{bmatrix} p_x \\ p_s \\ p_h \\ p_g \end{bmatrix}
    = - \begin{bmatrix} \nabla_x \mathcal{L} \\ -\mathcal{S} \Lambda_g e \\ h \\ g - s \end{bmatrix}
\end{equation}
where $\mathcal{L}$ the Lagrangian is defined in \eqref{eq:lag}, $\mathcal{S}$ and $\Lambda_g$ are as defined previously. 

NK method solves \eqref{eq:kkt0} using Krylov method, so it only need the matrix-vector product of the system matrix. The products are formed in a matrix-free way by solving two second-adjoint systems, for details, see~\cite{hicken:inexact2014, dener:idf2017} and the references therein.  However, while NK methods can avoid the cost associated with forming the explicit constraint Jacobian, the extension to inequality constraints faces several significant challenges. First, active-set and interior-point algorithms that make use of an explicit basis for the null space of the constraint Jacobian cannot be used, because such a basis requires the Jacobian to be explicitly available.  Second, the primal-dual saddle-point system raised in optimization is indefinite and ill-conditioned, making it difficult for iterative Krylov methods to converge to a sufficient tolerance, hurting the convergence rate of Newton's method. Third, dealing with nonconvex Hessian of the Lagrangian in the null space of the constraint Jacobian is also a non-trivial task. 



The solution to the optimization problem \eqref{eq:opt00x} is a saddle point for the Lagrangian \eqref{eq:lag}, as the optimal design point minimizes the Lagrangian, while the optimal multipliers maximizes the Lagrangian ~\cite{benzi2005numerical}. 
Note that the matrix in  \eqref{eq:kkt0} is nonsymmetric \footnote{By using a scaled slack block, $\tilde{p_s} = \mathcal{S}^{-1}e p_s $,  the system \eqref{eq:kkt0} can be turned into a symmetric one. We adopt the nonsymmetric version, as we think the preconditioner for this form has better numerical properties, see the section on preconditioner later}. 


\section{Krylov Iterative Solver}\label{2:krylov}
We use the flexible generalized minimal residual method,
FGMRES~\cite{Saad1993fgmres}, to solve~\eqref{eq:Newton}.  One advantage of
FGMRES, compared to most Krylov-based methods, is that is permits nonstationary
preconditioners that vary from one Krylov iteration to the next.  While we do
not take advantage of nonstationary preconditioners in this work, we have found
this flexibility invaluable in the solution of multidisciplinary optimization
problems~\cite{dener:idf2017, dener:2014}.

To find an approximate solution to \eqref{eq:Newton}, FGMRES orthonormalizes a
sequence of matrix-vector products using a generalized form of Arnoldi's
method~\cite{saad:2003}.  Starting with $v_{1} = b/\|b\|$, the generalized
Arnoldi's method produces the following relation at the $i$th iteration:
\begin{equation}\label{eq:arnoldi}
  (\nabla_q F) \mat{Z}_{i} = \mat{V}_{i+1} \bar{\mat{H}}_{i},
\end{equation}
where $\mat{V}_{i+1} \in \mathbb{R}^{N\times (i+1)}$ has orthogonal columns and
$\bar{\mat{H}}_{i} \in \mathbb{R}^{(i+1)\times i}$ is upper Hessenberg.  The
vectors in $\mat{Z}_{i} \in \mathbb{R}^{N\times i}$ form the subspace from which
the approximate solution to \eqref{eq:Newton} is drawn (see below).  These
vectors are related to the vectors in $\mat{V}_{i+1}$ by
\begin{equation*}
  z_{j} = P_j(v_{j}), \qquad \forall j = 1,2,\ldots,i,
\end{equation*}
where $P_j(\cdot)$ denotes the preconditioning operation at iteration $j$.    

The FGMRES solution is given by $x_{i} = \mat{Z}_{i} y_{i}$, where $y_{i}$ is
chosen to minimize the 2-norm of the residual, $r_i \equiv b -  (\nabla_q F) x_i = b -  (\nabla_q F)\mat{Z}_i y_i$, as
follows:
\begin{align*}
  y_{i} = \underset{y \in \mathbb{R}^i}{\textrm{argmin}}
  \lVert b -  (\nabla_q F) \mat{Z}_i y \| 
  &= \underset{y \in \mathbb{R}^i}{\textrm{argmin}}
  \lVert \mat{V}_{i+1} (\|b\| e_1 - \bar{\mat{H}}_{i} y \| \\
  &= \underset{y \in \mathbb{R}^i}{\textrm{argmin}}
  \lVert \|b\| e_1 - \bar{\mat{H}}_{i} y \|,
  \qquad\qquad\text{(since $\mat{V}_{i+1}^T \mat{V}_{i+1} = \mat{I}$)}
\end{align*}
where $e_{1} \in \mathbb{R}^{i+1}$ is the first column of the $(i+1)\times(i+1)$
identity. The minimization problem on the final line is inexpensive to solve in
practice, since $i$ is usually less than 100.

Like most Krylov iterative methods, the FGMRES algorithm described above does
not require the Jacobian $(\nabla_q F)$ explicitly.  From the user's
perspective, the algorithm only requires matrix-vector products and
preconditioning operations.  In this work, matrix-vector products involving
$(\nabla_q F)$ are computed using second-order adjoints~\cite{wang:1992,
  hicken:inexact2014}.  Briefly, each product with $(\nabla_q F)$ requires the
solution of two discretized linear PDEs: a linear forward problem and a linear
adjoint problem.  See \cite{dener:scitech2015} for further details regarding
second-order adjoints in the context of reduced-space problems with state
constraints.  The second required operation, preconditioning, is described in
the next subsection.




% a popular large-scale matrix-based active -set augmented-Lagrangian optimization method SNOPT.  

%\begin{equation}\label{eq:saddle}
%\mathcal{A} = \begin{bmatrix}
%A & B \\
%C & D
%\end{bmatrix}
%\end{equation}
%One type of Schur complement can be obtained by performing the following LDU decomposition 
%\begin{equation}\label{eq:saddle:ldu}
%\mathcal{A} = \begin{bmatrix}
%A & B \\
%C & D
%\end{bmatrix} = 
%\begin{bmatrix}
%I_p & BD^{-1} \\
%0 & I_q
%\end{bmatrix}
%\begin{bmatrix}
%A-BD^{-1}C  & 0 \\
%0  & D 
%\end{bmatrix}
%\begin{bmatrix}
%I_p  & 0 \\
%D^{-1}C  & I_q 
%\end{bmatrix}
%\end{equation}

% which is indefinite, poor spectral properties (ill-conditioned) 
% Direct solvers, however, are still the preferred method in optimization and other areas. Furthermore, direct methods are often used in the solution of subproblems, for example as part of a preconditioner solve.
%The preconditioners developed here are tailored for PDE-constrained optimization problem. 
% matrix-vector products with A can be performed efficiently
% approximate its action on a vector with (nearly) linear complexity,
% the convergence of the iterates to the optimal solution of problem
% gain efficient, save on storage
%words from the paper ~\cite{benzi2005numerical} Benzi block preconditioners, 
% As the iterates approaches the solution,  the entries in A tend to zero, or infinity, KKT matrix ill-conditioned
% the norm of the inverse Schur complement goes to infinity

% Schur complement reduction 
% null space methods, the null space of the constraint Jacobian, the column of Z span the null space of constraint Jacobian
% popular in optimization, projection of the problem onto the constraint set; 


% \begin{table}
% \caption[This is the Caption for Table 1]
%            {This is the Caption for Table 1\cite{thisbook}}
%  Note entry in [] for list of tables; you don't want the citation in the LOT.
% \begin{center}
% \begin{tabular}{lll}
% Here's       & an          & example  \\
% of           & a           & table    \\
% floated      & with        & the      \\
% \verb+table+ & environment & command.
%\end{tabular}
% \end{center}
% \end{table}
 
 
% \subsection{This is a Subsection Heading} 
 
% This is a sentence to take up space and look like text.
% This is a sentence to take up space \cite{anotherbook}.
% This is a sentence to take up space and look like text.
 
% \subsubsection{This is a Subsubsection Heading}
% This is a sentence to take up space and look like text.
% This is a sentence to take up space and look like text.
% Text before the footnote.\footnote{Here's the text of the footnote.}
% Text after the footnote. 
% This is a sentence to take up space and look like text.

%%% Local Variables: 
%%% mode: latex
%%% TeX-master: t
%%% End: 

%The formula \eqref{eq:kkt0} takes the same form as the classical interior point method in Chapter 19 in Nocedal's book \cite{Nocedal2006NO}, which will be reviewed below. 

%\section{Review on Interior Point Method  }
%The difficulty in extending the Newton Krylov methods to handle inequality constraints, to solve \eqref{eq:opt00x} lies in the nonlinear complementarity condition: for each inequality constraint, either the slack or Lagrangian multiplier is strictly zero if we assume strict complementarity is satisfied at the solution. The slack has to be non-negative to guarantee feasibility of the inequality constraints, and the multipliers has to be non-positive in respect to the property of a local minimization point following the formula convention. For inequality constraints that are active at the solution, slack variable is zero and the multiplier is negative, while for inactive inequality constraints, slack variable is positive and the multiplier is negative. Therefore, the complementarity condition, in combined with the sign requirement on slack and multipliers, contains information on optimal active set of the inequality constraints at the solution.   
%
%Currently there are two most powerful algorithms for general nonlinear constrained problems: active-set SQP methods and interior point methods \cite{Nocedal2006NO}. Determining the inequality constraint sets that are active at the solution is the main challenge facing active-set methods. Especially when the number of inequality constraints is large, the method may need many iterations to locate the active-set of inequality constraints. While for interior point methods, there are two varieties based on globalization strategies: Newton-Lagrangian line-search and trust-region SQP on the barrier problems. The former is more for illustration purpose, and the latter is actually implemented in practical optimization software libraries IPOPT \cite{W�chter2006}, and KNITRO \cite{Byrd:1999:IPA:588897.589167}. 
%
%The trust-region SQP method builds a quadratic model on the barrier formulation, employs direct linear algebra, uses explicit constraint Jacobians to first compute the multipliers that deliver minimum linearized constraint violations, then compute the design and slack update steps that minimize the quadratic model. In both subproblems, a trust region bound is imposed on the design and slack components, with the slack variable scaled properly to prevent it away from the nonnegative bound. A proper merit function mimic the quadratic objective function is used to estimate the quality of the steps and control the trust-region radius for next iteration. 
%
%To handle nonlinearities and nonconvexities, regularization terms can be added to the Hessian block and the equality constraint Jacobian on the diagonal of the KKT matrix. The proper amount of regularization is computed at each iteration by trial and error such that the inertia of the regularized KKT matrix is $(n+m, l+m, 0)$, under which condition the total Hessian block of design and slack will be positive definite on the null space of the combined constraint matrix, therefore the resulted Newton step will be a guaranteed descent direction for a large class of merit functions. 
%
%Using the proper barrier parameter $\mu$ updating strategy is crucial to the performance of interior point methods: A slowly decreasing $\mu$ will result in large number of outer iterations, making the algorithm less efficient. While a quickly decreasing $\mu$ may make some slack or inequality multipliers approach zero prematurely, hurting the convergence. Some simple implementations of interior point methods use a constant fraction updating scheme, while some chooses the fraction value based on the recent iterations' progress towards the solution. Making the fraction value close to zero near the solution can yield a superlinear convergence rate. More robust strategies update $\mu$ based on the progress of the current complementarity products. Predictor strategy first calculates a predictor direction by setting $\mu=0$, then calculates the tentative complementarity product along this direction using the step size from fraction to boundary rule. The updating fraction value is based on the ratio of this tentative and current complementarity product. 
%
%
%The former Newton-Lagrangian line-search method solves a perturbed KKT system at each homotopy parameter, also called the barrier parameter $\mu$:
%
%\begin{equation}\label{eq:kkt1}
%\begin{aligned}
%\nabla f(x) + \lambda_h^T \nabla h(x) + \lambda_g^T \nabla g(x) &= 0 \\
%-\mathcal{S} \Lambda_g - \mu e &= 0\\
%h(x) &= 0 \\
%g(x) - s &= 0 \\
%s \geq 0, \quad &\lambda_g \leq 0 \\
%\end{aligned}
%\end{equation}
%The barrier parameter $\mu$, is a sequence of strictly positive numbers and converges to zero. The perturbed KKT system \ref{eq:kkt1} is solved for each $\mu$, and the solution trajectory converges to the KKT point of the original problem in the limit.  
%
%Newton's method is used to solve \ref{eq:kkt1} for each $\mu$, where each Newton step is as follows:
%\begin{equation}
%\begin{bmatrix} \mathcal{W} & 0 & \mathcal{A}_{h}^T & \mathcal{A}_{g}^T \\
%    0 & -\Lambda_g & 0 & -\mathcal{S} \\
%    \mathcal{A}_{h} & 0 & 0 & 0 \\
%    \mathcal{A}_{g} & -\mathcal{I} & 0 & 0 
%    \end{bmatrix}
%    \begin{bmatrix} p_x \\ p_s \\ p_h \\ p_g \end{bmatrix}
%    = -\begin{bmatrix} \nabla_x \mathcal{L} \\ -\mathcal{S} \Lambda_g - \mu e  \\ h(x) \\ g(x) - s \end{bmatrix}
%\end{equation}
%After the Newton step direction is computed, fraction to the boundary rule is applied to determine the maximum allowable step size to keep the slack and inequality multipliers away from the 0 bound. Then a backtracking line search is performed to find the step length that delivers sufficient decrease in the merit function or accepted by the filter. The barrier parameter is then updated for the next iteration. 
%
%There are potential drawbacks when using interior point methods for PDE-constrained optimization. For instance, to ensure progress towards global minimum, either trust-region or line-search globalization techniques have to be implemented. The former judges the quality of a computed step by calculating the merit function value and adjust the trust-region radius accordingly, while the latter computes the step-length along a step direction that satisfies the Wolfe condition. In either case, extra computation is needed. Dealing with nonconvex Hessian of the Lagrangian in the null space of the constraint Jacobian is also a non-trivial task; possible solutions include adding a proper regularization term to enforce a positive definite Hessian, see \cite{hicken:flecs2014} and Algorithm B.1 \cite{Nocedal2006NO}. Moreover, the saddle-point matrix raised in optimization is indefinite and ill-conditioned, making it difficult for iterative Krylov methods to converge. 


