%%%%%%%%%%%%%%%%%%%%%%%%%%%%%%%%%%%%%%%%%%%%%%%%%%%%%%%%%%%%%%%%%%% 
%                                                                 %
%                            CHAPTER ONE                          %
%                                                                 %
%%%%%%%%%%%%%%%%%%%%%%%%%%%%%%%%%%%%%%%%%%%%%%%%%%%%%%%%%%%%%%%%%%% 
 
\chapter{Introduction}
 
\section{Motivation}
The thesis is interested in solving engineering design optimization problems that are governed by Partial-Differential-Equations (PDEs). Such PDE-constrained optimization problems arise in many engineering applications including aerodynamic shape optimization \cite{lambe:2014,lyu2014aerodynamic, Zhang567303}, structural optimization \cite{DBLP:DeckelnickHJ17, lambe:2014, kennedy14}, and thermodynamic optimization \cite{chen1999finite,bejan2000thermodynamic,bejan2012thermodynamic}.  %


\begin{figure}[H]
\centering
\subfloat[Aero-Structure Optimization~\cite{as_opt}]{
  \includegraphics[clip,width=0.5\columnwidth]{./figs/chap1_intro/1_as.png}\label{fig:A}  %
}
\subfloat[Topology Optimization~\cite{topo_opt}]{ %
  \includegraphics[clip,width=0.5\columnwidth]{./figs/chap1_intro/1_topo.png}\label{fig:B}
}
\caption{Large-Scale PDE-constrained Optimization}
\label{fig:1_mot}
\end{figure}

In such applications, an optimization library is coupled with a PDE solver. The simulation problem solves the PDEs for state variables, e.g. the pressure distribution along the surface mesh node points on the wing in Figure~\ref{fig:A}, the displacement distribution among the solid mesh node points in Figure~\ref{fig:B}, at certain design variables. The performance of the physics system is evaluated in the form of objective function bounded by certain constraints, while the objective and the constraints depend on both the state variables and the design variables. The optimizer will choose a design point, inquire the PDE solver for state variables, and compute the functional values and the gradients of the objective and constraints at that point. The optimizer will process all the information and yield a better design point using mathematical algorithms. The process is repeated until certain criteria is met for optimality.  

The optimization framework is simple, and has been used successfully in many application fields. The difficulty in PDE-constrained optimization is that the PDE solution (or state equation), which itself is a complicated topic to which intensive effort has been devoted, is a subproblem of the optimization problem. During the optimization process, the optimizer will query the PDE solver many times for the state variables and gradients at different design points; in addition to that, the optimizer need to compute and store the gradients of the constraints and objective in order to determine the next design point. In large-scale engineering, the PDE solve and gradient evaluation can be very expensive even with the advancement in modern computer power. Moreover, storage of the necessary matrices can impose a heavy burden on computational resources.    

General-purpose optimization algorithms are unsuitable for large-scale PDE-constrained optimization problems, because their matrix-based approaches require the computation and storage of the total constraint Jacobians, which in turn would result in poor scalability performance with ever-increasing size of problems. In addition, the commonly used limited-memory quasi-Newton methods in general-purpose algorithms usually brings linear asymptotic convergence rates, which is not ideal for large-scale optimization problems. 

This thesis intends to address these challenges by proposing a scalable optimization algorithm specially for large-scale PDE-constrained design problems. 

\section{Solving PDE-constrained Optimization Problem}\label{sec:pde_mot}
A generic PDE-constrained optimization problem can be stated as
\begin{equation}\label{eq:gen1}
\begin{aligned}
\underset{x,u}{\text{min}} \quad &f(x, u) &\\
\text{subject to} \quad &  h(x,u) &= 0  \\
 &  g(x,u) &\geq 0  \\
\text{governed by} \quad &  \mathcal{F}(x, u) &= 0, \\
\end{aligned}
\end{equation}
where $x \in \mathbb{R}^n, u \in \mathbb{R}^v$ are the design and state vectors respectively, and $f: \mathbb{R}^n \rightarrow \mathbb{R}, h: \mathbb{R}^n \rightarrow \mathbb{R}^l, g:\mathbb{R}^n \rightarrow \mathbb{R}^m$ are the objective, equality and inequality constraints respectively. We assume that $f$, $h$ and $g$ have continuous second derivatives. 

The challenge in solving \eqref{eq:gen1} is that general-purpose gradient-based optimization algorithms \cite{Nocedal2006NO, Byrd:1999:IPA:588897.589167,gill:2002} require the total derivative of the objective and constraints with respect to the design variables, and each of these total derivatives requires the solution of an $v\times v$ linear system, \ie a discretized PDE.  For example, the total derivative of the $i$th constraint $g_i$ is given by
\begin{align}
\frac{d}{dx}\left(g_i\right) &= \frac{\partial}{\partial x}\left(g_i\right) + \psi ^T \left[ \frac{\partial}{\partial x}\mathcal{F} \right] \\
\intertext{where $\psi \in \mathbb{R}^v$ is the adjoint which is governed by the discretized PDE,}
\left[ \frac{\partial}{\partial u}\mathcal{F} \right]^T \psi &= - \frac{\partial}{\partial u}\left(g_i\right) 
\end{align}
which is the adjoint equation \cite{Jameson03aerodynamicshape}. The computational cost of solving the adjoint equation is equivalent to that of the PDEs governing equation. Clearly, if the number of constraints $m$ is sufficiently large, it will be prohibitively expensive to compute all the total derivatives for optimization as each one entails an adjoint. 

To solve \eqref{eq:gen1}, the Lagrangian formulation is:
\begin{equation}\label{eq:lag}
\mathcal{L}(x, u, \psi, s, \lambda_h, \lambda_g) = f(x,u) + \lambda_h^T h(x, u) + \lambda_g^T (g(x,u)-s) + \psi^T \mathcal{F}(x,u),
\end{equation} 
where $s \in \mathbb{R}^m$ is for transforming the inequality constraints into equality ones, and $\lambda_h \in  \mathbb{R}^l$,  $\lambda_g \in  \mathbb{R}^m$ are the Lagrangian multiplier vectors for equality and inequality constraints. 

The first-order optimality conditions, or the \textit{Karush-Kuhn-Tucker} (KKT) conditions for \eqref{eq:gen1} are derived by taking partial derivatives of 
\eqref{eq:lag} with respect to all the unknown variables,
\begin{equation}\label{eq:kktcond}
\begin{aligned}
\nabla_x \mathcal{L} &= \nabla_x f + \lambda_h^T \nabla_x h + \lambda_g^T \nabla_x g + \psi^T \nabla_x\mathcal{F} = 0, \\
\nabla_u \mathcal{L} &= \nabla_u f + \lambda_h^T \nabla_u h + \lambda_g^T \nabla_u g + \psi^T \nabla_u\mathcal{F} = 0, \\
\nabla_{\psi} \mathcal{L} &= \mathcal{F} = 0, \\
\nabla_{\lambda_h} \mathcal{L} &= h = 0, \\
\nabla_{\lambda_g} \mathcal{L} &= g - s = 0, \\
-\mathcal{S} \Lambda_g e &= 0,\\
s \geq 0, &\quad \lambda_g \leq 0. \\
\end{aligned}
\end{equation}

The big nonlinear systems of equations \eqref{eq:kktcond} can be solved in either the full space or the reduced space.  Full-space methods~\cite{DBLP:journals/siamsc/BirosG05,DBLP:journals/siamsc/BirosG05a,haber:2001} solve all the unknowns in \eqref{eq:kktcond} simultaneously. The resulted KKT system is large, more than two times the number of state variables, indefinite and ill-conditioned. During optimization iterations, the state equation $\mathcal{F} =0$ and adjoint equation $\nabla_u \mathcal{L} = 0$ do not need to be solved exactly, avoiding the computational expense of tightly converging the PDE and adjoint equation residuals; however, this is also a potential disadvantage in practical engineering problems, because, if the optimization fails to converge, the intermediate solution may not be feasible with respect to the physics.  Furthermore, for highly nonlinear PDEs, \eg gas dynamics with shocks and boundary layers, practitioners have developed specialized globalization strategies that may be difficult to take advantage of in general-purpose full-space optimization algorithms.

In contrast, reduced-space algorithms treat the states $u$ and the adjoints $\psi$ as implicit functions of the design variables through $\mathcal{F}(x,u(x)) = 0$, $\nabla_u \mathcal{L} = 0$. The reduced KKT condition is formulated as follows, 
\begin{equation}\label{eq:opt00x}
 \begin{gathered}
    F(x,s,\lambda_h, \lambda_g) \equiv 
    \begin{bmatrix}
\nabla_x f + \lambda_h^T \nabla_x h + \lambda_g^T \nabla_x g\\
-\mathcal{S} \Lambda_g e\\
h  \\
g - s 
\end{bmatrix} =0,\\
\text{subject to} \quad s_i \geq 0, \quad \text{and} \quad \lambda_{gi} \leq 0. \\
\end{gathered}
\end{equation}

where $s \in \mathbb{R}^m$ denotes the slack variables, $q \equiv (x^T, s^T,
\lambda_h^T, \lambda_g^T) \in \mathbb{R}^{N}$ is a vector of all the unknowns,
and $F :\mathbb{R}^{N} \rightarrow \mathbb{R}^{N}$ is the vector-valued residual
of the KKT conditions excluding the inequalities on $s$ and $\lambda_g$.  For
convenience, we have also introduced $e = [1,1,\ldots,1]^T$ and the diagonal
matrices
\begin{equation*}
  \mat{S} = \mydiag\left(s_1,s_2,\ldots,s_m\right),\qquad\text{and}\qquad
  \mat{\Lambda_g} = \mydiag\left(\lambda_{g1}, \lambda_{g2}, \ldots, \lambda_{gm}\right).
\end{equation*}


Therefore, the resulted KKT system is much smaller, and it can make use of existing PDE solvers and adjoint solvers, maintaining modularity. Reduced-space methods have been successfully implemented in unconstrained problems and IDF problems. The unconstrained version of \eqref{eq:opt00x} can be solved efficiently using a Newton-Krylov (NK) algorithm applied to the first-order optimality conditions; see, for example, \cite{akcelik:2006, Heinkenschloss:1999:IOA, hinze2010optimization,borzi:2011}. NK optimization algorithms have also shown promise for equality-constrained optimization in the reduced space, because they do not require the constraint Jacobian to be form explicitly and, thus, avoid the scaling issue described earlier.  For instance, \cite{dener:idf2017} applied a matrix-free NK algorithm to a class of equality-constrained optimization problems that arise in multidisciplinary design optimization and would otherwise be intractable with conventional matrix-based algorithms.

Motivated by its success in the unconstrained and equality-constrained cases, we would like to extend the Newton Krylov methodology to more general, inequality constrained problems, to solve \eqref{eq:opt00x}. 

\section{Inexact-Newton Methods}

An alternative to using conventional (matrix-based) optimization algorithms or
constraint aggregation is to apply inexact-Newton methods~\cite{dembo:1982},
also know as truncated-Newton methods in the optimization
literature~\cite{nash:2000} to solve \eqref{eq:opt00x}.  

With the exception of the bounds on $s$ and $\lambda_g$, the KKT conditions
\eqref{eq:opt00x} form a set of nonlinear algebraic equations, $F(q)=0$.  These
equations can be solved, in principle, using Newton iterations of the form
\begin{equation}
(\nabla_q F) \Delta q^{(k)} = -F(q^{(k)}), \label{eq:Newton}
\end{equation}
where $q^{(k)}$ is the solution at the $k$th iteration and $\Delta q^{(k)} =
q^{(k+1)} - q ^{(k)}$ is the solution update.  Solving~\eqref{eq:Newton} exactly
can be inefficient during early Newton iterates when the linear model is not a
good approximation to $F(q)=0$.  Instead, truncated- and inexact-Newton methods
find approximate solutions to \eqref{eq:Newton} that, for example, satisfy the
inexact-Newton condition
\begin{equation}
  \left\| (\nabla_q F) \Delta q^{(k)} + F(q^{(k)}) \right\| \leq \eta_k \left\| F(q^{(k)}) \right\|, \label{eq:inexact_Newton}
\end{equation}
for some parameter $\eta_k \in (0,1)$.

There has been considerable success applying inexact-Newton methods to
unconstrained optimization problems; see \cite{nash:2000} and the references
therein.  On the other hand, inexact-Newton methods for general (nonconvex)
constrained problems are much less common.  Some notable exceptions include the
efforts by Byrd and colleagues~\cite{byrd:2008, byrd:2010} and by Heinkenschloss
and Ridzal~\cite{heinkenschloss:2014}; however, these algorithms make
assumptions regarding the structure of the problem that favor full-space
formulations, and our experience applying them to reduced-space PDE-constrainted
optimization has been disappointing.

Applying Newton's method to \eqref{eq:opt00x}, the KKT system, also called primal-dual system is obtained:
\begin{equation}\label{eq:kkt0}
\begin{bmatrix} \nabla_{xx} \mathcal{L} & 0 & \mathcal{A}_{h}^T & \mathcal{A}_{g}^T \\
    0 & -\Lambda_g & 0 & -\mathcal{S} \\
    \mathcal{A}_{h} & 0 & 0 & 0 \\
    \mathcal{A}_{g} & -\mathcal{I} & 0 & 0 
    \end{bmatrix}
    \begin{bmatrix} p_x \\ p_s \\ p_h \\ p_g \end{bmatrix}
    = - \begin{bmatrix} \nabla_x \mathcal{L} \\ -\mathcal{S} \Lambda_g e \\ h \\ g - s \end{bmatrix}
\end{equation}
where $\mathcal{L}$ the Lagrangian is defined in \eqref{eq:lag}, $\mathcal{S}$ and $\Lambda_g$ are as defined previously. 

NK method solves \eqref{eq:kkt0} using Krylov method, so it only need the matrix-vector product of the system matrix. The products are formed in a matrix-free way by solving two second-adjoint systems, for details, see~\cite{hicken:inexact2014, dener:idf2017} and the references therein.  However, while NK methods can avoid the cost associated with forming the explicit constraint Jacobian, the extension to inequality constraints faces several significant challenges. First, active-set and interior-point algorithms that make use of an explicit basis for the null space of the constraint Jacobian cannot be used, because such a basis requires the Jacobian to be explicitly available.  Second, the primal-dual saddle-point system raised in optimization is indefinite and ill-conditioned, making it difficult for iterative Krylov methods to converge to a sufficient tolerance, hurting the convergence rate of Newton's method. Third, dealing with nonconvex Hessian of the Lagrangian in the null space of the constraint Jacobian is also a non-trivial task. 

\section{Preconditioners for Optimization}
As is well known, Krylov iterative methods suffer from slow convergence rate when the spectral property of the system matrix is not favorable, e.g. the ratio of the largest to the smallest singular value is huge ($10^5$ to $10^9$). In such cases, a preconditioner that transforms the linear system into another one with favorable spectral properties is necessary for fast convergence of the iterative methods.  For saddle point systems, standard preconditioners like incomplete factorizations are often ineffective, as the indefiniteness of the system, or equivalently the zeros on the diagonal of the matrix would cause the preconditioner to breakdown in the pivoting operation steps \cite{saad:2003}. In addition, incomplete factorization involves direct operations to the entries of the matrix, which is not preferable as we do not have the matrix entries available in matrix-free method. One popular type of solver or preconditioner used in optimization is close to the null space approach, which involves computing the null space of constraint Jacobians, and finding the step update in that null space that would reduce the objective function values. The null space approach is not ideal either, as computing the column orthonormal vectors that span the null space involves operations to the matrix entries. In this thesis, a matrix-free preconditioner that makes use of the matrix vector products with the system matrix is proposed. 

The solution to the optimization problem \eqref{eq:opt00x} is a saddle point for the Lagrangian \eqref{eq:lag}, as the optimal design point minimizes the Lagrangian, while the optimal multipliers maximizes the Lagrangian ~\cite{benzi2005numerical}. 
Note that the matrix in  \eqref{eq:kkt0} is nonsymmetric \footnote{By using a scaled slack block, $\tilde{p_s} = \mathcal{S}^{-1}e p_s $,  the system \eqref{eq:kkt0} can be turned into a symmetric one. We adopt the nonsymmetric version, as we think the preconditioner for this form has better numerical properties, see the section on preconditioner later}. 
%Basic algebraic properties of the saddle point system like invertibility and conditioning in optimization can be analyzed through the Schur complement factorization form. Suppose the saddle point system is notated as


\section{Thesis Objective and Outline}

The objective of this thesis is to develop an efficient inexact-Newton algorithm that is
suitable for reduced-space PDE-constrained optimization problems of the
form~\eqref{eq:gen1}.  This goal faces two significant challenges, which
we seek to address in this work.
\begin{description}
\item[Nonconvexity:] The system $F(q) =0$ does not distinguish between different
  types of stationary points, so a basic Newton's method may converge to local
  maximizers or saddle points.  Conventional optimization algorithms often
  project onto the null-space of the (active) constraint Jacobian to detect
  directions of negative curvature and avoid undesirable stationary points, but
  the null-space is not explicitly available for matrix-free inexact-Newton
  methods.
\item[Preconditioning:] The number of iterations necessary to satisfy the
  inexact Newton condition \eqref{eq:inexact_Newton} using, for example, a
  Krylov method is closely related to the condition number of the system.
  Unfortunately, it is well known that the primal-dual, or KKT, matrix $\nabla_q
  F$ is indefinite and highly ill-conditioned.  A preconditioner is needed that
  is inexpensive to form, factor, and store.  A general-purpose, inexpensive
  preconditioner is especially difficult to find in the reduced-space context,
  since approximations to $\nabla_q F$ are not readily available as they are in
  the full-space.
\end{description}

The approach to addressing globalization is to introduce a homotopy map that
implicitly defines a solution curve that connects the solution to an easy
problem to the solution of the desired problem.  We then use a
predictor-corrector algorithm to follow the curve from the easy to the desired
solution.  A related globalization is used in \cite{Perez2009homotopy} in the
context of a trust-region managed sequential approximate optimization.  To
address the conditioning of the KKT matrix, we propose a low-rank approximation
of the Schur complement that is constructed by applying a few iterations of the
Lanczos method with approximate adjoints.

The following contents are organized as follows: 
\begin{itemize}
\item Chapter 2 reviews homotopy-based globalization and the homotopy map adopted in this work. Then it describes the predictor-corrector path-following algorithm that traces the homotopy zero curve. The proposed new optimization algorithm is called Homotopy RSNK method. 

\item Chapter 3 begins by reviewing Krylov iterative methods, but the focus is the proposed matrix-free preconditioner, firstly for solving inequality constrained problems, then for solving problems with both equality and inequality constraints. 
  
\item Chapter 4 begins with briefly introducing the optimization environment Kona, a state-of-the-art optimization algorithm SNOPT against which the performance of the new method is compared, and the parameter settings for the new method.  Then the optimization algorithm and preconditioners are tested on two analytical problems: 1) An indefinite problem to examine the algorithm's ability to bypass local maximum points. 2) A constructed quadratic problem with linear inequality constraints to investigate the scalability performance of the algorithm and effectiveness of the inequality-only preconditioner. 

\item Chapter 5 tests the Homotopy RSNK method, the inequality-only and mixed equality and inequality preconditioners on a subset of the CUTEr test problems. This Chapter briefly introduces CUTEr problems, the CUTEr problem classification and the Kona-CUTEr interface. Then it presents a table of numerical examples results using the new method and preconditioner, SNOPT.   
  
\item Chapter 6 applies the algorithm and preconditioner on a mass minimization stress-constrained structural problem. The results and the convergence history of the algorithm are compared with that of SNOPT.
  
\item Chapter 7 the new algorithm and preconditioner is applied on an aerodynamic shape optimization problem. The results and the convergence history of the algorithm are compared with that of SNOPT.

\item Chapter 8 provides conclusions and some suggestions for future works. 
\end{itemize}


% a popular large-scale matrix-based active -set augmented-Lagrangian optimization method SNOPT.  

%\begin{equation}\label{eq:saddle}
%\mathcal{A} = \begin{bmatrix}
%A & B \\
%C & D
%\end{bmatrix}
%\end{equation}
%One type of Schur complement can be obtained by performing the following LDU decomposition 
%\begin{equation}\label{eq:saddle:ldu}
%\mathcal{A} = \begin{bmatrix}
%A & B \\
%C & D
%\end{bmatrix} = 
%\begin{bmatrix}
%I_p & BD^{-1} \\
%0 & I_q
%\end{bmatrix}
%\begin{bmatrix}
%A-BD^{-1}C  & 0 \\
%0  & D 
%\end{bmatrix}
%\begin{bmatrix}
%I_p  & 0 \\
%D^{-1}C  & I_q 
%\end{bmatrix}
%\end{equation}

% which is indefinite, poor spectral properties (ill-conditioned) 
% Direct solvers, however, are still the preferred method in optimization and other areas. Furthermore, direct methods are often used in the solution of subproblems, for example as part of a preconditioner solve.
%The preconditioners developed here are tailored for PDE-constrained optimization problem. 
% matrix-vector products with A can be performed efficiently
% approximate its action on a vector with (nearly) linear complexity,
% the convergence of the iterates to the optimal solution of problem
% gain efficient, save on storage
%words from the paper ~\cite{benzi2005numerical} Benzi block preconditioners, 
% As the iterates approaches the solution,  the entries in A tend to zero, or infinity, KKT matrix ill-conditioned
% the norm of the inverse Schur complement goes to infinity

% Schur complement reduction 
% null space methods, the null space of the constraint Jacobian, the column of Z span the null space of constraint Jacobian
% popular in optimization, projection of the problem onto the constraint set; 


% \begin{table}
% \caption[This is the Caption for Table 1]
%            {This is the Caption for Table 1\cite{thisbook}}
%  Note entry in [] for list of tables; you don't want the citation in the LOT.
% \begin{center}
% \begin{tabular}{lll}
% Here's       & an          & example  \\
% of           & a           & table    \\
% floated      & with        & the      \\
% \verb+table+ & environment & command.
%\end{tabular}
% \end{center}
% \end{table}
 
 
% \subsection{This is a Subsection Heading} 
 
% This is a sentence to take up space and look like text.
% This is a sentence to take up space \cite{anotherbook}.
% This is a sentence to take up space and look like text.
 
% \subsubsection{This is a Subsubsection Heading}
% This is a sentence to take up space and look like text.
% This is a sentence to take up space and look like text.
% Text before the footnote.\footnote{Here's the text of the footnote.}
% Text after the footnote. 
% This is a sentence to take up space and look like text.

%%% Local Variables: 
%%% mode: latex
%%% TeX-master: t
%%% End: 

%The formula \eqref{eq:kkt0} takes the same form as the classical interior point method in Chapter 19 in Nocedal's book \cite{Nocedal2006NO}, which will be reviewed below. 

%\section{Review on Interior Point Method  }
%The difficulty in extending the Newton Krylov methods to handle inequality constraints, to solve \eqref{eq:opt00x} lies in the nonlinear complementarity condition: for each inequality constraint, either the slack or Lagrangian multiplier is strictly zero if we assume strict complementarity is satisfied at the solution. The slack has to be non-negative to guarantee feasibility of the inequality constraints, and the multipliers has to be non-positive in respect to the property of a local minimization point following the formula convention. For inequality constraints that are active at the solution, slack variable is zero and the multiplier is negative, while for inactive inequality constraints, slack variable is positive and the multiplier is negative. Therefore, the complementarity condition, in combined with the sign requirement on slack and multipliers, contains information on optimal active set of the inequality constraints at the solution.   
%
%Currently there are two most powerful algorithms for general nonlinear constrained problems: active-set SQP methods and interior point methods \cite{Nocedal2006NO}. Determining the inequality constraint sets that are active at the solution is the main challenge facing active-set methods. Especially when the number of inequality constraints is large, the method may need many iterations to locate the active-set of inequality constraints. While for interior point methods, there are two varieties based on globalization strategies: Newton-Lagrangian line-search and trust-region SQP on the barrier problems. The former is more for illustration purpose, and the latter is actually implemented in practical optimization software libraries IPOPT \cite{W�chter2006}, and KNITRO \cite{Byrd:1999:IPA:588897.589167}. 
%
%The trust-region SQP method builds a quadratic model on the barrier formulation, employs direct linear algebra, uses explicit constraint Jacobians to first compute the multipliers that deliver minimum linearized constraint violations, then compute the design and slack update steps that minimize the quadratic model. In both subproblems, a trust region bound is imposed on the design and slack components, with the slack variable scaled properly to prevent it away from the nonnegative bound. A proper merit function mimic the quadratic objective function is used to estimate the quality of the steps and control the trust-region radius for next iteration. 
%
%To handle nonlinearities and nonconvexities, regularization terms can be added to the Hessian block and the equality constraint Jacobian on the diagonal of the KKT matrix. The proper amount of regularization is computed at each iteration by trial and error such that the inertia of the regularized KKT matrix is $(n+m, l+m, 0)$, under which condition the total Hessian block of design and slack will be positive definite on the null space of the combined constraint matrix, therefore the resulted Newton step will be a guaranteed descent direction for a large class of merit functions. 
%
%Using the proper barrier parameter $\mu$ updating strategy is crucial to the performance of interior point methods: A slowly decreasing $\mu$ will result in large number of outer iterations, making the algorithm less efficient. While a quickly decreasing $\mu$ may make some slack or inequality multipliers approach zero prematurely, hurting the convergence. Some simple implementations of interior point methods use a constant fraction updating scheme, while some chooses the fraction value based on the recent iterations' progress towards the solution. Making the fraction value close to zero near the solution can yield a superlinear convergence rate. More robust strategies update $\mu$ based on the progress of the current complementarity products. Predictor strategy first calculates a predictor direction by setting $\mu=0$, then calculates the tentative complementarity product along this direction using the step size from fraction to boundary rule. The updating fraction value is based on the ratio of this tentative and current complementarity product. 
%
%
%The former Newton-Lagrangian line-search method solves a perturbed KKT system at each homotopy parameter, also called the barrier parameter $\mu$:
%
%\begin{equation}\label{eq:kkt1}
%\begin{aligned}
%\nabla f(x) + \lambda_h^T \nabla h(x) + \lambda_g^T \nabla g(x) &= 0 \\
%-\mathcal{S} \Lambda_g - \mu e &= 0\\
%h(x) &= 0 \\
%g(x) - s &= 0 \\
%s \geq 0, \quad &\lambda_g \leq 0 \\
%\end{aligned}
%\end{equation}
%The barrier parameter $\mu$, is a sequence of strictly positive numbers and converges to zero. The perturbed KKT system \ref{eq:kkt1} is solved for each $\mu$, and the solution trajectory converges to the KKT point of the original problem in the limit.  
%
%Newton's method is used to solve \ref{eq:kkt1} for each $\mu$, where each Newton step is as follows:
%\begin{equation}
%\begin{bmatrix} \mathcal{W} & 0 & \mathcal{A}_{h}^T & \mathcal{A}_{g}^T \\
%    0 & -\Lambda_g & 0 & -\mathcal{S} \\
%    \mathcal{A}_{h} & 0 & 0 & 0 \\
%    \mathcal{A}_{g} & -\mathcal{I} & 0 & 0 
%    \end{bmatrix}
%    \begin{bmatrix} p_x \\ p_s \\ p_h \\ p_g \end{bmatrix}
%    = -\begin{bmatrix} \nabla_x \mathcal{L} \\ -\mathcal{S} \Lambda_g - \mu e  \\ h(x) \\ g(x) - s \end{bmatrix}
%\end{equation}
%After the Newton step direction is computed, fraction to the boundary rule is applied to determine the maximum allowable step size to keep the slack and inequality multipliers away from the 0 bound. Then a backtracking line search is performed to find the step length that delivers sufficient decrease in the merit function or accepted by the filter. The barrier parameter is then updated for the next iteration. 
%
%There are potential drawbacks when using interior point methods for PDE-constrained optimization. For instance, to ensure progress towards global minimum, either trust-region or line-search globalization techniques have to be implemented. The former judges the quality of a computed step by calculating the merit function value and adjust the trust-region radius accordingly, while the latter computes the step-length along a step direction that satisfies the Wolfe condition. In either case, extra computation is needed. Dealing with nonconvex Hessian of the Lagrangian in the null space of the constraint Jacobian is also a non-trivial task; possible solutions include adding a proper regularization term to enforce a positive definite Hessian, see \cite{hicken:flecs2014} and Algorithm B.1 \cite{Nocedal2006NO}. Moreover, the saddle-point matrix raised in optimization is indefinite and ill-conditioned, making it difficult for iterative Krylov methods to converge. 


