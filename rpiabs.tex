%%%%%%%%%%%%%%%%%%%%%%%%%%%%%%%%%%%%%%%%%%%%%%%%%%%%%%%%%%%%%%%%%%% 
%                                                                 %
%                            ABSTRACT                             %
%                                                                 %
%%%%%%%%%%%%%%%%%%%%%%%%%%%%%%%%%%%%%%%%%%%%%%%%%%%%%%%%%%%%%%%%%%% 
 
\specialhead{ABSTRACT}
 In this thesis, a Reduced-Space Newton-Krylov method with Homotopy globalization technique is presented to solve PDE-constrained optimization problems.  
 
In the context of PDE-constrained optimization problems, 
%besides the design variables, the state variables also come into play. 
the objective and some of the constraint functions would depend on both the design and state variables. The  
total derivatives of the state-based objective and constraints necessitate an adjoint solve for each, which can be as expensive as the PDE solves. In the presence of many state-based constraints, it may not be practical to compute the constraint Jacobian explicitly. Besides, PDE-constrained problems can be very large, and 
%the dimension of PDE-constrained problems are usually very large, 
 storing the Jacobian matrices puts a heavy load on the computer memory.  A matrix-free Newton-Krylov optimization algorithm avoids these problems but presents additional challenges related to globalization, preconditioning, and inequality constraint handling abilities. 

To address these challenges, this thesis uses a globally convergent homotopy continuation approach to solve the first-order optimality conditions using reduced-space Newton-Krylov methods. It uses a predictor-corrector algorithm to trace the solution curve of the homotopy map that gathers the KKT condition and a homotopy regularization term. During the tracing process, the predictor and corrector linear systems are solved loosely by Krylov iterative methods. The matrix-vector products of the KKT matrix are approximated using second-order adjoints. This way, the computational cost of calculating the constraint Jacobians and memory cost of storing them are saved.  To cope with the poorly conditioned KKT system, matrix-free preconditioners based on Lanczos SVD approximation are proposed to accelerate the convergence of the Krylov iterative solvers.  

The proposed algorithm and preconditions are tested on a variety of problems. To verify its globalization ability, an inequality constrained sphere problem is first tested to check whether the new method can bypass local maximum points. Then a constructed high-dimensional quadratic indefinite problem with randomly generated non-convex pattern and random starting points is exercised; the statistical performance of 1000 cases are presented. The scaling performance of the algorithm and the effectiveness of the preconditioner is studied on a series of constructed quadratic problems with increasing dimensions, while the convergence results are compared with those of a popular matrix-based active-set optimization algorithm. The algorithm and preconditioners are also applied on a subset of CUTEr test problems to examine its accuracy and robustness. Finally, two PDE-constrained optimization problems are tested: the first one is a complex stress-constrained mass-minimization problem, which serves the case as many state-based constraints; the second one is an aerodynamic shape of the NASA Common Research Model wing based on Euler equations. In both cases, the results using the algorithm and preconditioners from this work are compared with those using state-of-the-art matrix-based optimization algorithm SNOPT. 

%%%%%%%%%%%%%%%%%%%%%%% 
