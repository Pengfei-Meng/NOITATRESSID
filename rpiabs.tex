%%%%%%%%%%%%%%%%%%%%%%%%%%%%%%%%%%%%%%%%%%%%%%%%%%%%%%%%%%%%%%%%%%% 
%                                                                 %
%                            ABSTRACT                             %
%                                                                 %
%%%%%%%%%%%%%%%%%%%%%%%%%%%%%%%%%%%%%%%%%%%%%%%%%%%%%%%%%%%%%%%%%%% 
 
\specialhead{ABSTRACT}
This thesis present a matrix-free method for partial differential equation (PDE)
constrained optimization problems formulated in the reduced space.  When many
state-based constraints are present in the reduced-space formulation, the
constraint Jacobian can become prohibitively expensive to compute explicitly,
because each constraint gradient requires the solution of a distinct adjoint
PDE.  This leads many practitioners to use constraint aggregation, which can
produce overly conservative solutions.  To avoid conservative solutions as well
as the expense of forming the constraint Jacobian, we adopt a matrix-free
inexact-Krylov optimization framework.  This choice introduces additional
challenges related to globalization and preconditioning.  To address
globlization, the proposed method uses a homotopy continuation approach and a
predictor-corrector algorithm to trace the solution curve.  The predictor and
corrector linear systems are solved using a Krylov iterative method with the
necessary matrix-vector products evaluated via second-order adjoints.  To cope
with the poorly conditioned primal-dual system, a matrix-free preconditioner is
proposed that uses a low-rank approximation of the Schur complement of the
primal-dual matrix; the low-rank approximation is constructed using a fixed
number of iterations of the Lanczos method. The algorithm is verified using
analytical problems, a subset of CUTEr problems, 
a stress-constrained mass minimization problem, and an aerodynamic shape optimization problem.  
The method shows promising performance relative to a state-of-the-art
matrix-based active-set algorithm, particularly for large numbers of design
variables.

%%%%%%%%%%%  Version 1  %%%%%%%%%%%% 
%In the context of PDE-constrained optimization problems, 
%the objective and the constraint functions often depend on both the design and state variables. 
%Gradient-based optimization needs the objective gradient and the constraint Jacobians. 
%The total derivatives of state-based objectives and constraints necessitate an adjoint variable, the solution of which can be as expensive as solving the primal state PDE. In the presence of many state-based constraints, it may not be practical to compute the constraint Jacobian explicitly. In addition, the number of constraints and variables present in some problems can make storage of the constraint Jacobian impractical. In this thesis, a Reduced-Space Newton-Krylov method with Homotopy globalization technique is presented to solve PDE-constrained optimization problems.  
% 
% A matrix-free Newton-Krylov optimization algorithm avoids explicit construction and storage of the Jacobian,
% but presents additional challenges related to globalization, preconditioning, and inequality constraint handling abilities. To address these challenges, this thesis uses a globally convergent homotopy continuation approach to solve the first-order optimality conditions using reduced-space Newton-Krylov methods. It uses a predictor-corrector algorithm to trace the solution curve of the homotopy map that combines the KKT conditions and the homotopy regularization term. During the tracing process, the predictor and corrector linear systems are solved inexactly by Krylov iterative methods. The matrix-vector products of the KKT matrix are approximated using second-order adjoints. To cope with the poorly conditioned KKT system, matrix-free preconditioners based on Lanczos SVD approximation are proposed to accelerate the convergence of the Krylov iterative solver.  
%
%The proposed algorithms are verified and benchmarked on a variety of problems. In particular, the optimization algorithm is applied to a subset of CUTEr test problems to examine its accuracy and robustness, and it is used to solve two PDE-constrained optimization problems. The optimization algorithm proposed in this work is compared 
%with a state-of-the-art matrix-based optimization algorithm.

%%%%%%%%%%%%%%%%%%%%%%% 
