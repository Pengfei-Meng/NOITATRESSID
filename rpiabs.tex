%%%%%%%%%%%%%%%%%%%%%%%%%%%%%%%%%%%%%%%%%%%%%%%%%%%%%%%%%%%%%%%%%%% 
%                                                                 %
%                            ABSTRACT                             %
%                                                                 %
%%%%%%%%%%%%%%%%%%%%%%%%%%%%%%%%%%%%%%%%%%%%%%%%%%%%%%%%%%%%%%%%%%% 
 
\specialhead{ABSTRACT}
 
The ever-growing capacity of computers are empowering numerical analysts to do simulations on ever-larger physics systems governed by Partial-Differential-Equations(PDEs). Accordingly, PDE-constrained optimization problems pose greater challenges as the optimization systems are larger and more complicated, especially when many state-based constraints are present. Current matrix-based optimization algorithms are defied for such problems due to memory requirements and computing cost. This thesis presents a matrix-free method for PDE constrained optimization problems which uses Krylov iterative method as the kernel solver, together with preconditioners for accelerating its convergence rate. The optimization algorithm is tested on analytical problems, a complex stress-constrained mass-minimization problem, and an aerodynamic shape optimization problem. The accuracy and scalability of the method is studied and compared with a state-of-the-art matrix-based optimization algorithm, particularly for large numbers of design variables and many constraints. 

For PDE-constrained optimization problem, the cost is dominated by PDE solves for evaluating different design points and adjoint solves for computing the total derivative of the state-based objective and constraints. In presence of many state-based constraints, it may not be practical to compute the constraint Jacobian explicitly.  This leads many practitioners to use constraint aggregation, which can produce overly conservative results.  A matrix-free Newton-Krylov optimization algorithm avoids these problems, but presents additional challenges related to globalization, preconditioning, and inequality constraint handling abilities. To address the former challenges, the proposed method uses a globally convergent homotopy continuation approach that uses a predictor-corrector algorithm to trace the solution curve. During the tracing process, the predictor and corrector phases are solved approximately to one or two digit accuracy using Krylov iterative method to keep a low per-iteration cost. The necessary matrix-vector products are approximated using second-order adjoints. To cope with the poorly conditioned primal-dual system, respective preconditioners are proposed to accelerate the convergence of the Krylov iterative solvers.  

The algorithm possesses moderate non-convex handling capacity, which is verified on a class of analytical non-convex problems. Its scaling performance is studied on a series of constructed quadratic problems, in comparison with the results from a popular matrix-based active-set optimization algorithm. A practical complex stress-constrained mass-minimization problem, which serves as the case with many state-based constraints, is investigated for the algorithm's accuracy and robustness, with the results compared with that from the popular algorithm. Finally, the algorithm is used for optimizing the aerodynamic shape of the NASA Common Research Model wing in 3D Euler, the optimal design is benchmarked with previous literatures, and the scalability performance studied.   