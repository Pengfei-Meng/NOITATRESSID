%%%%%%%%%%%%%%%%%%%%%%%%%%%%%%%%%%%%%%%%%%%%%%%%%%%%%%%%%%%%%%%%%%%
%                                                                 %
%                            APPENDICES                           %
%                                                                 %
%%%%%%%%%%%%%%%%%%%%%%%%%%%%%%%%%%%%%%%%%%%%%%%%%%%%%%%%%%%%%%%%%%%
 
\appendix    % This command is used only once!
%\addcontentsline{toc}{chapter}{APPENDICES}             %toc entry  or:
\addtocontents{toc}{\parindent0pt\vskip12pt APPENDICES} %toc entry, no page #

\chapter{THIS IS AN APPENDIX}

\begin{landscape}     % ADD CUTER PROBLEM SETTNGS!  refer to this table for all problems involved!!
\begin{longtable}{ l c c c c c c c }
%  \begin{center}
    \caption{Parameters used in the test problems \label{tab:param}} \\
    \textbf{Parameters} & $\textbf{Default}$     & $\textbf{Range}$ &  $\textbf{Sphere}$    &   $\textbf{Non-convex}$ 
    & $ \textbf{Quadratic} $   & \textbf{Structural}    &  \textbf{ASO} \\ \hline
    %\rule{0ex}{3ex}%
    \multicolumn{8}{ l }{Predictor-Corrector Algorithm} \\   %  $\delta_{\text{targ}}$ and $\phi_{\text{targ}}$. 
    \hline    
    $K_{\max}$             	&  100     & $\geq$100   & 500      & 100 	 &  100       &    100  & 100    \\ 
     $J_{\max}$  		&   2         & $\geq$2         & 2    & 2          & 2           &      2     & 2   \\
    $\epsilon_F$ 	           		&  1e-6     & [1e-8, 1e-3]    & 1e-7   & 1e-7 	 & 1e-7       &    1e-4   & 1e-3  \\ 
       $\tau$    		&   0.1      & [0.1,0.5]	    & 0.1    & 0.1          & 0.1	 &     0.1   & 0.1  \\
    $\epsilon_H$    		&   0.1      & [0.1,0.5]	    & 0.1       & 0.1          & 0.1	 &     0.1   & 0.1    \\
   % $\epsilon_{\mu} $   & 1e-9     & [1e-10, 1e-6]   &   &   1e-9  & 1e-9  & 1e-6  &  \\   
    \textbf{$\alpha_0$}             &  0.05     & $\geq$0.01    & 0.05      & 0.05	 & [40,60,80,100,120]  &  0.05  & 0.05 \\
    $\delta_{\text{targ}}$      &  1.0	& [1.0,10]       & 1	    & 10		 & 10    &   1.0   & 10	  \\
    $\phi^{\circ}_{\text{targ}}$   & 10.0	& [5.0,50] 	     & 10     & 10		 & 20    &   10    & 20    \\
    $\zeta_{\max}$ 		        &  50		& [10,50]	& 50       & 50		 & 50    	 &   50   & 50  \\
    $\zeta_{\min}$ 		        &  0.5	& 0.5		& 0.001       & 0.001	 & 0.001    &   0.5  & 0.001 \\
    $\Delta \mu_{\max}$		        &  -5e-4	& [-5e-4, -5e-1] & -5e-4 & -5e-4	 & -5e-4     &  -5e-4 & -5e-4 \\  
    $\Delta \mu_{\min}$		        &  -0.9	& -0.9 	& -0.9	       & -0.9		 & -0.9       &  -0.9  & -0.9	  \\
    $s_0$                           & $\mathbf{e}$     &   $>$ 0  & $g(x_0)$  &    5$\mathbf{e}$    &  10$\mathbf{e}$   &  $g(x_0)$  & $g(x_0)$  \\ 
    $\tau_s$                      & 1e-6    & 1e-6    &  1e-6 &  1e-6   &  1e-6    & 1e-6  &  1e-6 \\
    \hline
    \multicolumn{8}{ l }{Preconditioner} \\ 
    \hline    
    $\mathbf{{n_{\mat{\Sigma}}}}$    & 5	       & $\geq$2	  & -    &  -	         &  2       &  [20,80,320]   & 20 \\
    $\beta$				& 1.0	       & $>$0        & -      & -    &  -      &  0.1 & -  \\
    $N_{\text{bfgs}}$		& 10	       & [1, 20]		& - 	&  -   	 &  10	& -     &  20    \\
    $\mu_e$			& -1	       & [0, 1] 	& -	& -	         &  -	        & 1e-3  & 1e-3\\
    $\Sigma_e$			& 1 	       & [0, 1]      & -           & -		& -		& 1e-3 & - \\
    \hline
    \multicolumn{8}{ l }{Krylov Iterative Solver} \\ 
    \hline       
    $n_k$		& 20        & [10,30]              & 20	 & 20	 &  20       &  20   & 20  \\
    $\epsilon_{\text{krylov}}$		& 1e-2     & [0, 0.1]    &1e-2       	&1e-2	 &  1e-2    &  1e-4  &1e-2\\
    \hline
%  \end{center}
\end{longtable}    
\end{landscape}



\section{Brief overview of the CUTEr test problems}\label{sec:cuternm}
For each test problem, CUTEr provides access to the objective function and the constraint functions, as well as their derivatives.
% The objective value is a real scalar: 
%\begin{equation*}
%y = f(x) 
%\end{equation*}
%where  $x \in \mathbb{R}^n, y \in \mathbb{R}$. $x_i$ is the i-th component of x. 

The constraints include bound constraints on the variable and other types of constraints. The variable $x$ are subject to simple bounds:
\begin{equation*}
\text{bl}_i \leq x_i \leq \text{bu}_i  
\end{equation*}
where $\text{bl}_i$ and $\text{bu}_i$ are the lower and upper bound on $x_i$. When there is no lower or upper bound on $x_i$, then $\text{bl}_i = -1e20$ or $\text{bu}_i = 1e20$. 

With the exception of the bound constraints, all remaining 
constraints are gathered in a single vector-valued function $c(x) \in \mathbb{R}^m$, which is then bounded as follows:   
\begin{equation*}
\text{cl}_i \leq c_i(x) \leq \text{cu}_i  
\end{equation*}
where one of $\text{cl}_i$ or $\text{cu}_i$ is always around $10^{20}$. This means that the inequality constraint must take just one of 
the following two forms on a given problem:
\begin{equation*}
c_i(x) \geq 0   \quad  \text{or}  \quad  c_i(x) \leq 0.
\end{equation*}
For equality constraints $\text{cl}_i$ and $\text{cu}_i$ are both equal to 0. There is also a bool vector, $\text{Equatn} \in \mathbb{R}^m$, indicating whether a constraint is an equality constraint or not. 

It is possible to instruct CUTEr to order equality constraints before inequality constraints, or linear constraints before nonlinear constraints, by reordering the components of $c(x)$. Likewise, components of $x$ can be reordered such that nonlinear variables appear before linear ones. 

In addition, CUTEr can also output 
the Lagrangian function value, the gradient of the objective function and the Lagrangian, the Jacobian matrix and the Hessian matrix of the constraints. 

\section{Kona-CUTEr Interface}\label{sec:konacut}
By using a Python interface to CUTEr~\cite{cuter_python}, one can access the test problems in Python environment and build the Kona-CUTEr interface. As described in~\cite{dener:scitech2016}, a Kona solver interface essentially asks for matrix-vector products with the PDE-Jacobian (if any),  
 and matrix-vector and vector-matrix products with the Jacobians of the equality and inequality constraints. 
The explicit Jacobian matrices from CUTEr can be readily used to calculate their product with an arbitrary incoming vector for Kona.  CUTEr problems are still valuable to verify Kona's accuracy, as well as its capability to overcome other numerical difficulties, including non-convex problems. 

%As discussed in chapter \ref{sec:kona_mv}, Kona might not be able to solve CUTEr problems as fast as conventional optimization methods like SNOPT, because the Jacobians are explicitly available in CUTEr problems.  


\section{Problem Classification}\label{sec:cuter_clas}
Each problem in CUTEr is classified following the Hock and Schittkowski scheme~\cite{cuterScheme} by the string: 
\begin{equation*}
X \ X \ X \ r \ - \ X \ X \ - \ n \ - \ m
\end{equation*}

The first character defines the problem objective function type, with the following options: 
\begin{itemize}  \itemsep -8pt 
\item \textbf{U}: undefined,
\item  \textbf{C}: constant, 
\item \textbf{L}: linear, 
\item \textbf{Q}: quadratic, 
\item  \textbf{S}: sum of squares, 
\item  \textbf{O}: none of the above. 
\end{itemize}

The second character defines the constraint function type, with the options: 
\begin{itemize}  \itemsep -8pt 
\item  \textbf{U}: unconstrained,
\item  \textbf{X}: fixed variables, 
\item  \textbf{B}: bounds on the variables,
\item  \textbf{N}: adjacency matrix of a linear network,
\item  \textbf{L}: linear, 
\item  \textbf{Q}: quadratic,
\item  \textbf{O}: more general constraints. 
\end{itemize}

The third character shows the smoothness of the problem, with the option of:
\begin{itemize}  \itemsep -8pt 
\item  \textbf{R} : the problem is regular, and its first and second derivatives exist and continuous everywhere,
\item  \textbf{I}: the problem is irregular.
\end{itemize}

The third character \textbf{r} holds an integer among 0, 1 and 2, indicating the highest derivatives provided analytically.

The first character after the hyphen indicates the origin of the problem, with the option of:
\begin{itemize}  \itemsep -8pt 
\item  \textbf{A}: the problem is academic, mainly used by researchers to test algorithms; 
\item  \textbf{M}: the problem is a modeling exercise, with the solution not used in practical application; 
\item  \textbf{R}: the solution of the problem has been used in a real application.  
\end{itemize}

The next character has an option of: 
\begin{itemize}  \itemsep -8pt 
\item \textbf{Y}: the problem contains internal variables, 
\item \textbf{N}: the problem does not contain internal variables. 
\end{itemize}

The last two characters have the following options:
\begin{itemize}  \itemsep -8pt 
\item \textbf{n} - \textbf{m}: the number of variables and constraints (fixed variables and bound constraints excluded), 
\item \textbf{V} - \textbf{V}: an integer chosen by the user among the given list of fixed numbers.
\end{itemize}


\section{A Section Heading}

This is how equations are numbered in an appendix:
\begin{equation}
x^2 + y^2 = z^2
\end{equation} 
This is a sentence to take up space and look like text.

\chapter{THIS IS ANOTHER APPENDIX} 
This is a sentence to take up space and look like text.
